\documentclass{article}
\usepackage{graphicx} % Required for inserting images
\usepackage[margin=1in]{geometry}
\usepackage{fancyhdr}
\usepackage{listings}
\usepackage{courier}
\usepackage{multicol}
\usepackage{anyfontsize}

% Header and Footer
\pagestyle{fancy}
\fancyhf{}
\fancyhead[RO,LE]{\textbf{Escuela de código PILARES}}
\fancyfoot[LO,CE]{Carlos Ignacio Padilla Herrera}
\fancyfoot[RO,RE]{\thepage}

% Listings settings
\lstset{
    basicstyle=\fontsize{8}{11}\selectfont\ttfamily,
    breaklines=true,
    captionpos=b,
    numbers=left,
    numberstyle=\tiny,
    stepnumber=1,
    numbersep=5pt,
    showspaces=false,
    showstringspaces=false,
    showtabs=false,
    tabsize=2,
    columns=fullflexible,
    frame=single, % Add a frame around the code
}

% Title Page
\title{Lenguajes Populares en el Mundo del Desarrollo Web}
\author{Carlos Ignacio Padilla Herrera}
\date{03 de junio de 2024}

\begin{document}

\maketitle

\section*{Lenguajes Populares en el Mundo del Desarrollo Web}

\subsection*{JavaScript}
\begin{itemize}
    \item \textbf{Diseñado para brindar dinamismo a las páginas web}
    \item \textbf{Validar formularios y muchas cosas más para mejorar la experiencia del usuario}
    \item \textbf{Manipular el Document Object Model (DOM)}
    \item \textbf{Crear aspectos visuales dinámicos}
    \item \textbf{Desarrollo de juegos 2D y 3D}
    \item \textbf{Desarrollo de extensiones de navegador}
    \item \textbf{Web apps y aplicaciones móviles}
    \item \textbf{Desarrollo de servidores}
\end{itemize}

\section*{Características Nuevas y Mejoras en JavaScript}
\begin{itemize}
    \item \textbf{Nuevas características}
    \item \textbf{Mejor rendimiento}
\end{itemize}

\section*{Metalenguajes}
\begin{itemize}
    \item \textbf{HTML}
    \item \textbf{CSS}
\end{itemize}

\section*{Lenguajes de Programación}
\begin{itemize}
    \item \textbf{C}
    \item \textbf{C++}
    \item \textbf{VBA}
    \item \textbf{Y otros}
\end{itemize}

\section*{Lenguajes de Propósito Específico}
\begin{itemize}
    \item \textbf{Orientados a una tarea concreta y específica}
    \item \textbf{Facilitan operaciones en páginas web}
\end{itemize}

\section*{Intervalos de Desarrollo}
\begin{enumerate}
    \item \textbf{Intervalo de diseño de páginas web dinámicas}
    \item \textbf{Intervalo de mejora de la experiencia del usuario}
    \item \textbf{Intervalo de desarrollo de juegos y aplicaciones}
\end{enumerate}

\end{document}
