\documentclass{article}
\usepackage[utf8]{inputenc}
\usepackage{graphicx}
\title{Taller de Bases de Datos}
\author{Carlos Padilla}
\date{\today}

\begin{document}

\maketitle

\section{Introducción}
Este documento presenta un mapa conceptual de tres gestores de bases de datos destacados, describiendo sus características principales, así como sus ventajas y desventajas.

\section{Gestores de Bases de Datos}
\subsection{MySQL}
\textbf{Características Principales:}
\begin{itemize}
    \item Sistema de gestión de bases de datos relacional (RDBMS) de código abierto.
    \item Usa SQL para el manejo de datos.
    \item Compatible con una gran variedad de plataformas.
\end{itemize}
\textbf{Ventajas:}
\begin{itemize}
    \item Alta velocidad y fiabilidad.
    \item Amplio soporte comunitario.
    \item Bajo costo de propiedad.
\end{itemize}
\textbf{Desventajas:}
\begin{itemize}
    \item Menos eficiente en sistemas muy grandes.
    \item Falta de soporte para algunos estándares SQL avanzados.
\end{itemize}
\includegraphics[width=\linewidth]{mysql.png}

\subsection{PostgreSQL}
\textbf{Características Principales:}
\begin{itemize}
    \item Sistema de gestión de bases de datos relacional de código abierto.
    \item Ofrece soporte completo para transacciones ACID.
    \item Soporta un amplio conjunto de tipos de datos.
\end{itemize}
\textbf{Ventajas:}
\begin{itemize}
    \item Soporte para consultas complejas y extensibilidad.
    \item Alta conformidad con los estándares SQL.
    \item Personalizable y con una comunidad activa.
\end{itemize}
\textbf{Desventajas:}
\begin{itemize}
    \item Más complejo de administrar.
    \item Puede ser más lento en operaciones masivas de escritura.
\end{itemize}
\includegraphics[width=\linewidth]{postgresql.png}

\subsection{MongoDB}
\textbf{Características Principales:}
\begin{itemize}
    \item Base de datos NoSQL orientada a documentos.
    \item Almacena datos en formatos de tipo JSON con esquemas dinámicos.
    \item Soporta replicación y sharding de forma nativa.
\end{itemize}
\textbf{Ventajas:}
\begin{itemize}
    \item Altamente flexible.
    \item Escalabilidad horizontal.
    \item Rápido procesamiento de datos.
\end{itemize}
\textbf{Desventajas:}
\begin{itemize}
    \item Menor consistencia de datos.
    \item No adecuado para transacciones complejas multi-tabla con garantías ACID.
    \item Requiere más espacio de almacenamiento.
\end{itemize}
\includegraphics[width=\linewidth]{mongodb.png}

\end{document}

