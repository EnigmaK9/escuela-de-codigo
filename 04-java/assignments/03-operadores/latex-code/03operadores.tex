\documentclass{article}
\usepackage{graphicx} % Required for inserting images
\usepackage{pdflscape}
\usepackage[margin=1in]{geometry}
\usepackage{fancyhdr}
\usepackage{listings}
\usepackage{courier}
\usepackage{multicol}
\usepackage{anyfontsize}
\usepackage{tcolorbox} % Required for the border around the entire document

% Header and Footer
\pagestyle{fancy}
\fancyhf{}
\fancyhead[RO,LE]{\textbf{Escuela de código PILARES}}
\fancyfoot[LO,CE]{Carlos Ignacio Padilla Herrera}
\fancyfoot[RO,RE]{\thepage}

% Listings settings
\lstset{
    basicstyle=\fontsize{8}{11}\selectfont\ttfamily,
    breaklines=true,
    captionpos=b,
    numbers=left,
    numberstyle=\tiny,
    stepnumber=1,
    numbersep=5pt,
    showspaces=false,
    showstringspaces=false,
    showtabs=false,
    tabsize=2,
    columns=fullflexible,
    frame=single, % Add a frame around the code
}

% Title Page
\title{03. Operadores}
\author{Carlos Ignacio Padilla Herrera}
\date{03 de junio de 2024}

\begin{document}

% Add a border around the entire document
\tcbset{
    colframe=black, % Frame color
    colback=white, % Background color
    width=\textwidth, % Box width
    enlarge left by=\dimexpr\hoffset+\oddsidemargin+1in\relax,
    enlarge right by=\dimexpr\hoffset+\oddsidemargin+1in\relax,
    top=0pt, bottom=0pt, left=0pt, right=0pt, % Margins inside the box
    boxrule=1pt, % Border thickness
}

\tcbstartrecording

% Cover Page
\begin{titlepage}
    \centering
    \vspace*{2cm}

    \Huge
    \textbf{03. Operadores}

    \vspace{1.5cm}

    \LARGE
    \textbf{Padilla Herrera Carlos Ignacio}

    \vspace{0.5cm}

    \Large
    \textbf{Folio: 794DR03}

    \vspace{1.5cm}

    \LARGE
    \textbf{Escuela de Código PILARES}


    \LARGE
    \textbf{Java Entre Semana G0224}

    \vspace{0.5cm}

    \Large
    Fecha: 03 de junio de 2024

    \vfill
\end{titlepage}

\maketitle

\section*{03. Operadores}
\textbf{Objetivo:} Verificar el dominio teórico y técnico de los principios básicos de operadores en Java mediante preguntas de opción múltiple y ejercicios para desarrollar su código.

\textbf{Indicaciones:} Realiza los programas que se te solicitan, una vez realizado sube el archivo de tu actividad.

\section*{Actividades}

\subsection*{Programa 1. (2 puntos)}
Supón que intentamos construir algunas de las funciones de una cuenta bancaria. Considera el siguiente código:

\begin{lstlisting}[language=Java, caption={Código base del programa CuentaBancaria.java}]
public class CuentaBancaria {

    public static void main(String[] args){
        double saldo = 1000.75;
        double cantidadARetirar = 250;
    }
}
\end{lstlisting}

\begin{itemize}
    \item Crea una nueva variable double llamada \textit{saldoActualizado} y almacene \textit{saldo} con \textit{cantidadARetirar} restada de él.
    \item Ahora, imagina que has decidido dividir tu saldo en 3 partes iguales para dárselo a tus tres mejores amigos. Crea una variable double llamada \textit{cantidadParaCadaAmigo} que contenga \textit{saldoActualizado} dividido por 3.
    \item Si cada uno de tus amigos quiere comprar un boleto para un concierto con el dinero que les diste y las entradas cuestan 250, crea una variable de tipo boolean llamado \textit{puedeComprarTicket} y configúralo de tal manera que arroje si \textit{cantidadParaCadaAmigo} tiene lo suficiente para comprar una entrada para el concierto.
    \item Usa \textit{System.out.println()} para imprimir \textit{puedeComprarTicket}.
    \item Usa \textit{+} y \textit{System.out.println()} para imprimir con el valor de \textit{cantidadParaCadaAmigo} en lugar de \textit{<cantidadParaCadaAmigo>}. Le di a cada amigo \textit{<cantidadParaCadaAmigo>}...
    \item Toma captura de pantalla del código completo y del programa compilado.
\end{itemize}

\begin{lstlisting}[language=Java, caption={Código del programa CuentaBancaria.java}]
/*
 * Click nbfs://nbhost/SystemFileSystem/Templates/Licenses/license-default.txt to change this license
 */

package com.mycompany.cuentabancaria;

/**
 *
 * @author enigma
 */
public class CuentaBancaria {

    public static void main(String[] args) {
        // Define el saldo inicial en la cuenta bancaria
        double saldo = 1000.75;

        // Define la cantidad a retirar de la cuenta bancaria
        double cantidadARetirar = 250;

        // Calcula el saldo actualizado despues del retiro
        double saldoActualizado = saldo - cantidadARetirar;

        // Divide el saldo actualizado entre tres amigos
        double cantidadParaCadaAmigo = saldoActualizado / 3;

        // Verifica si cada amigo puede comprar un ticket de 250
        boolean puedeComprarTicket = cantidadParaCadaAmigo >= 250;

        // Imprime si cada amigo puede comprar un ticket
        System.out.println("Puede comprar ticket?: " + puedeComprarTicket);

        // Imprime la cantidad que se le dio a cada amigo
        System.out.println("Le di a cada amigo " + cantidadParaCadaAmigo + "...");
    }
}


\end{lstlisting}

\newpage

\begin{landscape}
    \begin{figure}[h]
        \centering
        \includegraphics[width=\linewidth]{img/cuentaBancaria.png}
        \caption{Captura del programa CuentaBancaria.java}
        \label{fig:captura}
    \end{figure}
\end{landscape}

\end{document}

\tcbset{record}
