\documentclass{article}
\usepackage[utf8]{inputenc}
\usepackage{graphicx}
\usepackage{geometry}
\geometry{a4paper} % Set the page size to A4
\usepackage{listings} % Package for including code in the document

\title{Práctica 06: Creación de una base de datos}
\author{Carlos I. Padilla Herrera}
\date{12 de mayo de 2024}

\lstset{frame=single, % Adds a frame around the code
        basicstyle=\small\ttfamily, % Use a small, true type font
        language=SQL, % SQL syntax highlighting
        showstringspaces=false} % Don't mark spaces in strings

\begin{document}

\maketitle

\section*{Información del Taller}
\textbf{Taller:} Escuela de código PILARES: Base de datos entre semana \\
\textbf{Número de la práctica:} 6 \\
\textbf{Nombre completo:} Padilla Herrera Carlos Ignacio \\
\textbf{Folio:} 794DR02 \\
\textbf{Fecha:} 12 de mayo de 2024 \\
\textbf{Actividad:} Se describen a continuación las instrucciones para la creación de tablas y inserción de datos en una base de datos SQL. Para detalles adicionales, referirse al enlace del curso.

\newpage

\section*{Código SQL para la creación de tablas y inserción de datos}

\begin{lstlisting}
-- Creating tables
create table manufacturer (
    code int(10) primary key,
    name varchar(100)
);

create table product (
    code int(10) primary key,
    name varchar(100),
    price double,
    manufacturer_code int(10),
    foreign key (manufacturer_code) references manufacturer(code)
);

-- Inserting manufacturers
insert into manufacturer (code, name) values
(1, 'Asus'),
(2, 'Lenovo'),
(3, 'Hewlett-Packard'),
(4, 'Samsung'),
(5, 'Seagate'),
(6, 'Crucial'),
(7, 'Gigabyte'),
(8, 'Huawei'),
(9, 'Xiaomi');

-- Inserting products
insert into product (code, name, price, manufacturer_code) values
(1, 'Disco Duro SATA3 1TB', 86.99, 5),
(2, 'Memoria RAM DDR4 8GB', 120, 6),
(3, 'Disco SSD 1TB', 150.99, 4),
(4, 'GeForce GTX 1050 Ti', 185, 7),
(5, 'GeForce GTX 1080 Xtreme', 755, 6),
(6, 'Monitor 24 LED Full HD', 202, 1),
(7, 'Monitor 27 LED Full HD', 245.99, 1),
(8, 'Portátil Yoga 520', 559, 2),
(9, 'Portátil Ideapad 320', 444, 2),
(10, 'Impresora HP Deskjet 3720, 59.99, 3)
(11, 'Impresora HP Laserjet Pro M26nw', 180, 3);
\end{lstlisting}

\newpage % Inicia una nueva página

\section*{Captura de pantalla}

\begin{figure}[h!]
    \centering
    \includegraphics[width=\linewidth]{screenshot.png} % Asegúrate de poner el nombre correcto del archivo y la extensión
    \caption{Captura de pantalla de DB Fiddle}
\end{figure}


\end{document}
