\documentclass{article}
\usepackage[utf8]{inputenc}
\usepackage[spanish]{babel}
\usepackage{geometry}
\geometry{landscape} % Set the page orientation to landscape
\usepackage{tikz}
\usetikzlibrary{shapes, arrows.meta, positioning}
\usepackage{graphicx} % Required for including images
\usepackage{fancyhdr} % Required for header and footer management
\pagestyle{fancy}
\fancyhf{} % clear all header and footer fields

\title{EDC PILARES Práctica 05: Gestores de bases de datos}
\author{Carlos I. Padilla Herrera. Folio 794DR02}
\date{\today}

\begin{document}

\maketitle
\newpage % Start a new page for the conceptual map

\section{Mapa Conceptual}

\begin{center}
\begin{tikzpicture}[node distance=0.5cm, auto] % Adjusted node distance for better spacing
  \node (Gestores) [rectangle, draw=blue, thick, fill=blue!30, text width=5em, text centered, minimum height=3em] {Managers};

  % Database management nodes with increased spacing
  \node (mysql) [rectangle, draw=violet, thick, fill=violet!30, below left=5cm and 5cm of Gestores, text width=5em, text centered, minimum height=3em] {MySQL};
  \node (postgres) [rectangle, draw=olive, thick, fill=olive!30, below left=5cm and 2.5cm of Gestores, text width=5em, text centered, minimum height=3em] {PostgreSQL};
  \node (mongodb) [rectangle, draw=teal, thick, fill=teal!30, below=5cm of Gestores, text width=5em, text centered, minimum height=3em] {MongoDB};
  \node (oracle) [rectangle, draw=red, thick, fill=red!30, below right=5cm and 2.5cm of Gestores, text width=5em, text centered, minimum height=3em] {Oracle};
  \node (sqlserver) [rectangle, draw=black, thick, fill=gray!30, below right=5cm and 5cm of Gestores, text width=5em, text centered, minimum height=3em] {SQL Server};

  % Child nodes for MySQL
  \node (security) [rectangle, draw=red, thick, fill=red!30, below=of mysql, text width=5em, text centered, minimum height=3em, align=center] {Seguridad\\Integración de sistemas};
  \node (cluster) [rectangle, draw=orange, thick, fill=orange!30, below=of security, text width=6em, text centered, minimum height=3em, align=center] {Cluster\\Herramientas de administración};
  \node (replication) [rectangle, draw=pink, thick, fill=pink!30, below=of cluster, text width=5em, text centered, minimum height=3em, align=center] {Replicación\\Estrategias};

  % Child nodes for PostgreSQL
  \node (performance) [rectangle, draw=cyan, thick, fill=cyan!30, below=of postgres, text width=5em, text centered, minimum height=3em, align=center] {Mejoras en \\desempeño};
  \node (backup) [rectangle, draw=purple, thick, fill=purple!30, below=of performance, text width=5em, text centered, minimum height=3em, align=center] {Soluciones de \\respaldo};
  \node (acid) [rectangle, draw=lime, thick, fill=lime!30, below=of backup, text width=6em, text centered, minimum height=3em, align=center] {Cumplimiento con\\ACID};

  % Child nodes for MongoDB
  \node (scalability) [rectangle, draw=maroon, thick, fill=maroon!30, below=of mongodb, text width=6em, text centered, minimum height=3em, align=center] {Carácteristicas de\\Escalabilidad};
  \node (nosql) [rectangle, draw=gray, thick, fill=gray!30, below=of scalability, text width=5em, text centered, minimum height=3em, align=center] {Modelos de datos\\NoSQL};
  \node (performanceM) [rectangle, draw=green, thick, fill=green!30, below=of nosql, text width=5em, text centered, minimum height=3em, align=center] {Monitoreo de \\desempeño};

  % Child nodes for Oracle
  \node (transaction) [rectangle, draw=orange, thick, fill=orange!30, below=of oracle, text width=6em, text centered, minimum height=3em, align=center] {Control de \\Transacciones};
  \node (enterprise) [rectangle, draw=pink, thick, fill=pink!30, below=of transaction, text width=6em, text centered, minimum height=3em, align=center] {Carácteristicas\\Empresariales};
  \node (cloud) [rectangle, draw=blue, thick, fill=blue!30, below=of enterprise, text width=5em, text centered, minimum height=3em, align=center] {Integración con \\Nube};

  % Child nodes for SQL Server
  \node (integration) [rectangle, draw=cyan, thick, fill=cyan!30, below=of sqlserver, text width=5em, text centered, minimum height=3em, align=center] {Servicios de \\Integración};
  \node (analytics) [rectangle, draw=purple, thick, fill=purple!30, below=of integration, text width=5em, text centered, minimum height=3em, align=center] {Análitica de \\Negocios};
  \node (securityS) [rectangle, draw=orange, thick, fill=orange!30, below=of analytics, text width=6em, text centered, minimum height=3em, align=center] {Carácteristicas de \\seguridad};

  % Images adjusted to not overlap with the last node
  \node (mysqlimg) [below=of replication, text width=5em, text centered, yshift=-1cm] {\includegraphics[width=1cm]{mysql.png}};
  \node (postgresimg) [below=of acid, text width=5em, text centered, yshift=-1cm] {\includegraphics[width=1cm]{postgresql.png}};
  \node (mongodbimg) [below=of performanceM, text width=5em, text centered, yshift=-1cm] {\includegraphics[width=1cm]{mongodb.png}};
  \node (oracleimg) [below=of cloud, text width=5em, text centered, yshift=-1cm] {\includegraphics[width=1cm]{oracle.png}};
  \node (sqlserverimg) [below=of securityS, text width=5em, text centered, yshift=-1cm] {\includegraphics[width=1cm]{sqlserver.png}};

  % Paths
  \draw[-] (Gestores) -- (mysql);
  \draw[-] (Gestores) -- (postgres);
  \draw[-] (Gestores) -- (mongodb);
  \draw[-] (Gestores) -- (oracle);
  \draw[-] (Gestores) -- (sqlserver);

  \draw[-] (mysql) -- (security);
  \draw[-] (security) -- (cluster);
  \draw[-] (cluster) -- (replication);

  \draw[-] (postgres) -- (performance);
  \draw[-] (performance) -- (backup);
  \draw[-] (backup) -- (acid);

  \draw[-] (mongodb) -- (scalability);
  \draw[-] (scalability) -- (nosql);
  \draw[-] (nosql) -- (performanceM);

  \draw[-] (oracle) -- (transaction);
  \draw[-] (transaction) -- (enterprise);
  \draw[-] (enterprise) -- (cloud);

  \draw[-] (sqlserver) -- (integration);
  \draw[-] (integration) -- (analytics);
  \draw[-] (analytics) -- (securityS);
\end{tikzpicture}
\end{center}

\newpage

\section{Descripción de Gestores de Bases de Datos}

\subsection{MySQL}
\textbf{Ventajas:}
\begin{itemize}
  \item Ampliamente utilizado en la industria, especialmente en aplicaciones web.
  \item Soporte robusto y una gran comunidad de usuarios.
\end{itemize}

\textbf{Desventajas:}
\begin{itemize}
  \item Escalabilidad vertical más que horizontal, puede ser un limitante en sistemas de gran tamaño.
  \item No es ideal para aplicaciones que requieren manejo extensivo de grandes datos no estructurados.
\end{itemize}

\subsection{PostgreSQL}
\textbf{Ventajas:}
\begin{itemize}
  \item Soporta un mayor volumen de tipos de datos que otros gestores de bases de datos SQL.
  \item Extensible, permite a los desarrolladores crear y usar sus propias funciones.
\end{itemize}

\textbf{Desventajas:}
\begin{itemize}
  \item Puede ser más complejo de administrar debido a su configuración y opciones avanzadas.
  \item Potencialmente menor rendimiento con cargas de trabajo muy grandes en comparación con algunas bases de datos NoSQL.
\end{itemize}

\subsection{MongoDB}
\textbf{Ventajas:}
\begin{itemize}
  \item Esquema de datos flexible, ideal para datos no estructurados y variados.
  \item Escalabilidad horizontal, facilita la distribución de datos en varios servidores.
\end{itemize}

\textbf{Desventajas:}
\begin{itemize}
  \item La consistencia eventual puede ser un problema para aplicaciones que requieren alta consistencia en tiempo real.
  \item Gestión de transacciones más compleja en comparación con los sistemas SQL tradicionales.
\end{itemize}

\subsection{Oracle}
\textbf{Ventajas:}
\begin{itemize}
  \item Alta durabilidad y rendimiento para aplicaciones empresariales críticas.
  \item Amplias características de seguridad y cumplimiento.
\end{itemize}

\textbf{Desventajas:}
\begin{itemize}
  \item Alto costo de licencias y mantenimiento.
  \item Complejidad en la configuración y la gestión para usuarios menos experimentados.
\end{itemize}

\subsection{SQL Server}
\textbf{Ventajas:}
\begin{itemize}
  \item Integración profunda con herramientas de Microsoft, ideal para entornos que utilizan otras tecnologías de Microsoft.
  \item Potentes herramientas de análisis y reporting integradas.
\end{itemize}

\textbf{Desventajas:}
\begin{itemize}
  \item Costo elevado comparado con soluciones de código abierto.
  \item Menor flexibilidad en comparación con bases de datos de código abierto.
\end{itemize}

\end{document}

